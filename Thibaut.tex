\chapter{Application : voiles}

\begin{multicols}{2}
\section{Introduction}
   Le kevlar est un matériau utilisé dans une grande variété de domaines.  Parmi les applications les plus connues il y a les gilets pare-balles, des parties d’ailes d’avion, la coque des bateaux,…  Comme application pour notre matériau nous avons décidé de nous intéresser à l’utilisation du Kevlar pour les voiles de voiliers et plus particulièrement les voiliers de compétition.  Grâce à ses propriétés physiques et chimiques incroyables, le Kevlar est toujours très largement utilisé dans les compétitions nautiques et cela, à juste titre.

\section{Le Kevlar, oui mais quel Kevlar?}
Tout d’abord, pourquoi se focaliser sur les voiliers de compétitions et non les voiliers de plaisance ?  Simplement parce que le Kevlar présente des caractéristiques formidables pour la performance mais il présente des gros désavantages pour des cotés plus pratiques et financiers.  Penchons-nous donc plus en détails sur propriétés de ce matériau.  Nous nous focaliserons surtout sur le Kevlar 49 et le Kevlar 29 pour notre application, en effet ces deux types sont principalement utilisé pour les voiles tandis que d’autre Kevlar tel que le 129 ne saurait être utilisé pour faire une voile compétitive.

Le Kevlar 49 est 50\% plus résistant que le Kevlar 29 mais est moins flexible.  En ce qui concerne la voile, un bon équilibre entre la résistance et la flexibilité est nécessaire, une voile trop flexible va se déformer plus vite et donc apporter une résistance moindre au vent et la propulsion du navire en sera affectée.  Au contraire, une voile trop peu flexible (Kevlar 129) va avoir une surface au vent moindre et donc une surface de voile plus petite.  Un bon équilibre entre les deux  va transformer plus directement la force du vent en vitesse au bateau.   Une voile trop flexible fasseye, c’est-à-dire que la voile n’est pas tendue et qu’il y a de la surface qui flotte comme du linge qui pend.  Cela affecte évidement capacité totale de propulsion.  Il y a peu, la société DuPont a créé un nouveau Kevlar appelé le Kevlar Edge.  C’est un Kevlar hybride qui allie les points forts du Kevlar 29 et 49.  Le Kevlar Edge est 25\% plus flexible que le Kevlar 49 mais garde une résistance égale,  cela allie les avantages du 29 et du 49. 

\section{Les avantages}
Le Kevlar a beaucoup de propriétés géniales pour la voile :  c’est un matériau extremement resistant   (5 fois le module de l’acier à poids égale) et très léger.  Sachant qu’en voile, il y a deux choses principales à prendre en compte pour la vitesse, à savoir la propulsion et la trainée.  En allégeant au maximum le poids du voilier, la trainée sera réduite et en ayant une bonne resistance et flexibilité de la voile, la propulsion sera efficace.  Le Kevlar a une très bonne resistance aux tractions, sachant que le vent gonfle les voiles, on peut dire qu’il n’y a que des efforts de tractions sur la voile et très peu de compressions.  Le Kevlar n’offrant pas de dilatation thermique notable, cela permet d’avoir une voile apportant les mêmes propriétés par tous les temps.  Le fait que le Kevlar est moins sensible aux milieux alcalins qu’en milieu acide, cela apporte un avantage énorme en compétition sur eau de mer.  Comme la mer est légèrement alcacline, la voile ne sera pas affectée par des éclaboussures ou autre mouillage.  Cela est n’est nullement négligeable lorsqu’on passe plusieurs semaine en mer. 

\section{Les désavantages}
Parmis les désavantages du Kevlar il y a le peu de resistance qu’offre ce matériau au rayonnement UV.  La couleur d’origine sable devient noir et la voile s’en trouve affectée. L’exposition au UV, au soleil donc, affaiblit leur souplesse et leur solidité.  La voile est aussi sensible au fluage, c’est-à-dire au pliage.  Le pliage endommage la voile et il faut donc des véhicules adaptés pour transporter les voiles qui peuvent prendre des dimensions assez grandes.  Même en les pliant, la voile peut garder une taille assez conséquente.  On peut aussi citer une reprise d’humidité importante (4\%) qui va alourdir quelque peu la voile quand celle-ci est mouillée et va perdre en résistance.  L’usinage du Kevlar est assez complexe et donc le coût de production va s’en trouver augmenté.


	\begin{center}
		\includegraphics[scale=0.5]{phkevlar.png}  
	\end{center}
	

\section{Voiles avec plusieurs matériaux}
Le Kevlar peut aussi être utilisé pour renforcer une voile avec ou sans Kevlar : pour les coutures ou simplement amener plus de résistance à un endroit voulu.  Il y a différentes façons de faire des voiles hybrides.
\subsection{Les voiles laminées}
Pour les voiles laminées, on utilise plusieurs matériaux pour avoir les avantages de chacun d’eux.
Le film est pris en sandwich entre deux couches de tissus parfois appelé taffetas. Le film procure une plus grande résistance à élongation (déformation de la voile) qu'un tissu, les deux tissus apportent une protection du film en améliorant la résistance à l'abrasion et aux déchirures. Les versions haut de gamme utilisent du Spectra tissé ou du taffetas de Kevlar. Dans de récentes évolutions, des fils d'aramide sont inclus dans le film. Parfois une seule couche de tissu (ou taffetas) au lieu de deux est mise pour soit des questions de cout ou de légèreté de voile.  On obtient donc un enchainement film-fibre-film. 
\subsection{Fabrication en 3DL} Pour fabriquer ce type de voile, on recourt à une technique appelée 3DL.  Tout d’abord on conceptionne la voile sur un ordinateur en 3D, un programme lit le plan et envoie les instructions à un robot qui fabrique sur mesure la base de la voile.  Des machines cousent les fils sur la base, ce qui donne la première couche.  On veille à mettre les mêmes tensions de fils partout pour que la voile ait une résistance uniforme.   On procède à un laminage (on colle les fils et le film) pour bien fixer les deux couches.   Ensuite on recouvre la dernière couche par un film qu’on va tendre et qu’on va fixer grâce à un grand sac sous vide qui va appliquer une très grande force pour seller les différentes parties de la voile.  Pour finir, on chauffe la voile avec une quantité de chaleur très étudiée pour fixer une dernière fois les différentes couches entre elles.
\subsection{Tissage}
On peut également tisser une voile avec du fil de kevlar et de carbone.  Près des anneaux, il faut une très grande résistance pour que la voile  ne se déchire pas.  Le kevlar intervient donc et il y aura plus de tissage avec une grande quantité de kevlar dans cette partie-là.

\end{multicols}
